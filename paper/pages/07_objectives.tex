% Objectives of the study
%\section{Objectives of the study}
%Objectives of our study

\subsection{General Objective}
To develop an AIoT visual sensory substitution hat device that converts real-time environment information through emulating human sensory processes into audio cues for the visually impaired people (VIP)

\subsection{Specific Objectives}
\begin{enumerate}
    % Software
    % \item To develop a prototype main system that converts environmental information to audio cues that includes:
    % \begin{enumerate}
    %     \item A photogrammetry (2D to 3D) system using dual OV5640 stereo cameras and Inertial Measurement Unit (IMU) to generate depth maps and point cloud calculatio
    %     \item Toggable modes to prioritize what information is cued into audio
    %     \item Allows insertion anywhere in the pipeline for more custom preferences
    % \end{enumerate}
    \item To develop a prototype system that includes:
    \begin{enumerate}
        \item A photogrammetry (2D to 3D) system using dual OV5640 stereo cameras and Inertial Measurement Unit (IMU) to convert environmental information through depth maps and point cloud calculation enabling real-time 3D reconstruction of environment into audio cues for navigation.
        \item A toggleable cue mode by hand gesture commands using (what did u use na program for hand gesture) to adjust perception focus, to prioritize specific spatial information into audio cues for navigation or to configure external services with the users’ preferences.
        \item A modular system that inserts custom module AI services using (what did u use External API?) such as object, face, and text recognition in real-time controlled by hand gesture commands, and location-based services using SIM868 GPS module to track route of the user.
    \end{enumerate}
    % Hardware
    % \item To develop the device to be portable and comfortable
    % \begin{enumerate}
    %     \item Uses battery instead of needing direct connection to power
    %     \item Can be charged through UPS instead of disposable batteries
    %     \item All packed into a hat to avoid putting strain on sensitive areas like eyes, nose, and ears
    % \end{enumerate}
    % AI Testing
    % \item To test the accuracy and speed of the AI models
    % \begin{enumerate}
    %     \item Determine the frames per seconds of AI or Image processes
    %     \item Compare depth predictions with actual depth
    %     \item Failure cases of Language Models fullfilling request
    % \end{enumerate}
    % User/Device Evaluation
    % \item To evaluate the user's comfort and acclimation towards the device
    % \begin{enumerate}
    %     \item Does the device become uncomfortable during long periods of usage?
    %     \item Does the main pipeline become subconscious to the user's experience?
    %     \item Can the user properly interpret information the device is providing them?
    % \end{enumerate}
    % Hardware and User
    % \item To develop the device to be portable and comfortable to the user
    % \begin{enumerate}
    %     \item The device is powered through UPS allowing portability and chargeable
    %     \item All packed into a hat to avoid putting strain on sensitive areas like eyes, nose, and ears
    %     \item Using the device becomes subconscious to the user's experience
    % \end{enumerate}
    \item To design a portable device that includes:
    \begin{enumerate}
        \item A rechargeable 5V/3A UPS module with 18650 lithium-ion batteries for cord-free operation via Type-C charging
        \item 3D-printed hat enclosure to distribute weight evenly and prevent strain on sensitive facial areas like eyes, nose, and ears.
    \end{enumerate}

    \item To test the functionality of the device system by:
    \begin{enumerate}
        \item Calculating the Error and Success Percentage performance of photogrammetry by comparing the generated depth maps and point clouds to ground truth measurements.
        \item Evaluating the Precision, Recall, and Accuracy of the toggleable cue mode hand gesture commands responsiveness to switching modes.
        \item Measuring the frame per second (FPS) of AI processing, error rates in recognition, and route accuracy of the modular system.
        \item Simulating full system functionality by combining photogrammetry, IMU, and modules to track overall FPS, audio cue fidelity, and failure cases, with qualitative feedback from the users using survey.
    \end{enumerate}

    % !User experience
    % \item To develop an interface that is not visually dependent nor over stimulating
    % \begin{enumerate}
    %     \item Command and instructions are derived from hand guestures through hand recognition
    %     \item The commands and instructions are easy to understand and navigate
    %     \item The audio output is controlled minimizing unessary information
    % \end{enumerate}
    % !Testing
    % \item To test if users develop unsubconscious visual senses of the environment
    % \begin{enumerate}
    %     \item They can navigate indoor through rooms/doors/furnitures
    %     \item They can detect movement and even catch trown objects
    % \end{enumerate}
\end{enumerate}