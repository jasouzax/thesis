% Conceptual Framework
\subsection{Conceptual Framework}

\begin{figure}[H]
    \centering
    \caption{IPO Model of the Conceptual Frame of the Study}
    \begin{plantuml}
        @startuml
        skinparam rectangle {
            BackgroundColor WhiteSmoke
            BorderColor Black
        }
        skinparam linetype ortho

        rectangle "<size:14><b>INPUT</b></size>\n\n<b>Hardware:</b>\n• Orange PI 2W\n• 5V/3A UPS module\n• OV5640 Camera\n• Binaural Earphone\n• SIM868 with GPS\n• MPU 6050\n\n<b>Software:</b>\n• GitHub\n• Microsoft Word\n• Linux\n• DenoJS\n• VSCode\n\n<b>Wireless Connectivity:</b>\n• 2/3G or Wi-Fi" as input #WhiteSmoke

        rectangle "<size:14><b>PROCESS</b></size>\n\n\n\n\n1. Problem Statement\n2. Literature Review\n   (RRL)\n3. Prototyping\n   1. Requirements\n   2. Designing\n   3. Gathering materials\n4. Development\n   5. Refinement\n   6. Implementation\n   7. Testing\n   8. Analysing\n\n\n" as process #WhiteSmoke

        rectangle "<size:14><b>OUTPUT</b></size>\n\n\n\n\nDevelopment of\nWearable Navigation Aid\nusing AIoT for the\nVisually Impaired\n\n\n\n\n\n\n\n\n\n" as output #WhiteSmoke

        rectangle "<size:14><b>FEEDBACK</b></size>" as feedback #WhiteSmoke

        input -right-> process
        process -right-> output
        output -down-> feedback
        feedback -left-> input
        @enduml
    \end{plantuml}
    \label{fig:ipo}
\end{figure}

%Figure \ref{fig:ipo} states the conceptual framework of our research in a form of an Input-Process-Output (IPO) model. We start with three kinds of required input such has Hardware for our Microcontroller, Sensors, Actuators, and SIM Module, Software for our Project management, Client-side device, and Server-side device, and the Wireless Connectivity used in our device which can be either 2/3G which is used on deployment doors, or Wi-Fi which is used deployment indoors or development. Our method of research mimics typical development cycle with addition to literature review and research, such has literature review, prototyping, designing, etc. By the end of each iterative development we are to expect to be closer to the end goal, with a feedback from either advisor or mentor to start the cycle again until our end goal of an wearable navigation aid is developed.


\todo{Explain Futher}
Figure \ref{fig:ipo} states the conceptual framework of the research in the form of an Input-Process-Output (IPO) model. Starting with three kinds of required input such as the hardware components Microcontroller, Sensors, Actuators, and SIM Module, as the Software for Project management, Client-side device, and Server-side device, and the Wireless Connectivity used on to the device which can be either 2/3G used for deployment doors, or Wi-Fi as used in deployment indoors or development. The method of the research mimics typical development cycles with addition to literature review and research, such as literature review, prototyping, designing, testing/evaluating, and implementation. By the end of each iterative development the researchers expects to be closer to the end goal, with feedback from users and professionals to start the cycle again until the end goal of a wearable navigation aid is developed.