\subsection{Ethical Considerations}

\begin{enumerate}
    \item \textbf{Informed Consent and Autonomy:} \\
        All participants will be informed of their rights, such as the right to voluntary consent, particularly during FGDs involving VIPs and during prototype testing. The researchers will secure approval of implementation and evaluation and provide an accessible format, such as audio instruction or a face-to-face explanation, to present the project's purpose, duration, recording methods, and data use to ensure the participants' understanding despite communication barriers. For professional consultations with experts, the researchers will confirm their willingness to participate, disclose any conflicts of interest, and clarify their voluntary role. Participants will be informed that they have the right to withdraw at any time without facing repercussions, which is crucial in group settings to mitigate peer pressure or to avoid participating during the testing of the project's functionality. Researchers will assess participants' capacity to consent, involving caregivers as needed, while preserving participants' IP rights.
    \item \textbf{Privacy and Confidentiality:} \\
        The researchers will practice data privacy and confidentiality when acquiring participants' personal experiences or device feedback data. The researchers will obtain consent, ensure data is anonymized, and securely store it to prevent unauthorized access, excluding participants' identifiable details from reports. In addition, during professional consultations, they often share information that must remain private, so the researchers will need confidentiality agreements to protect ideas and opinions. Participants will be informed that what they share in the group will stay within the group. The researchers will only collect the necessary data for AI models and get clear permission for any other uses. Finally, the data acquired will be protected by encryption, and access will be exclusive to authorized researchers.
    \item \textbf{Safety and Risk Assessment:} \\
    The researchers will focus on reducing and evaluating physical and mental harm. They will set up safety protocol checks and conduct pre-tests during ongoing evaluations of the navigation hat. This device has components such as batteries and sensors that might overheat, fail, or produce incorrect audio signals, which could lead to falls. During user testing, there will be emergency protocols to stop testing immediately and provide medical assistance if the device fails. Users may experience cognitive overload or sensory fatigue after prolonged device use. It is important to watch for adverse effects like disorientation or anxiety and to have psychological support available to ensure participants stay safe and well.
    \item \textbf{Beneficence and Non-Maleficence:} \\
        Minimize harm while maximizing benefits. The researchers will monitor the FGDs for emotional distress when discussing navigation challenges and provide support resources. Researchers will also ensure venues are accessible to prevent physical or psychological harm to VIPs. The discussion improves the study's quality through user feedback and enhances the device design for VIPs, but researchers must address the prototype's expectations and limitations to avoid false anticipations. Improve the AI process without biased advice. Promote equity by making technology accessible and affordable for various VIP groups to avoid exclusion. Evaluate the long-term impact on overall well-being and the risk of isolation due to decreased human interaction.
    \item \textbf{Justice and Inclusivity:} \\
        Fair representation will be ensured when selecting participants and consultants for FGDs and when evaluating the prototype, including people from different backgrounds, with varying levels of impairment, ages, genders, and local cultural views on disability in the Philippines. This helps prevent bias and respects cultural norms. Accessibility is also essential. The researchers will use tools like audio aids for visually impaired participants and ensure that logistical issues do not exclude anyone. Our goal is to benefit persons with disabilities (PWDs), especially VIPs, beyond researchers and institutions, and to promote equal participation in research outcomes.
    \item \textbf{Transparency and Integrity:} \\
        Transparency and integrity will clearly outline the study's objectives during focus group discussions (FGDs) and consultations to avoid misleading participants. The researchers will document how the input received influences the project, such as by refining the modular system. In addition, the researchers will seek neutral experts who can provide objective advice. Ethical practices will be followed, including appropriately attributing contributions while safeguarding participants' privacy to uphold research integrity.
    \item \textbf{AI and Technology Ethics:} \\
        Potential biases in photogrammetry models or audio cues can happen when the training data is not representative. This can lead to inaccuracies and require the researchers to conduct regular fairness checks. The researchers will clearly explain AI decision-making to participants, ensuring it aligns with human sensory experiences without being misleading. The researchers must also address dual-use concerns, such as using real-time tracking for surveillance, which requires safeguards to protect user privacy and prevent unauthorized data sharing.
    \item \textbf{Inclusion and Exclusive Criteria:} \\
        The study will exclude VIPs aged 17 years old and below (minors) and 60 years old and above (senior citizens) as respondents. Only those aged 18-59 years old will participate in the study's survey, testing, and evaluation of the prototype.
    \item \textbf{Regulatory and Institutional Compliance:} \\
        The researchers will submit plans for focus group discussions (FGD), consultations, and testing to the university's Institutional Review Board (IRB) or an advisor for approval. This is particularly important given the vulnerability of vulnerable populations (VIPs). The goal is to ensure compliance with ethical standards and legal frameworks, including the Philippines' Data Privacy Act, especially regarding recordings and information sharing. Same with testing the device, the researchers will ensure that sensor and camera data are anonymized and encrypted to comply with privacy laws, and establish emergency protocols for handling device failures during trials. Post-activity follow-ups, such as summaries of the FGD and consultation to build trust and allow feedback on the use of input. Researchers will disclose conflicts of interest, such as affiliations with tech companies, to ensure impartiality.        
\end{enumerate}
