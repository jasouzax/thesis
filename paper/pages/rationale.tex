% Rationale of the study
\subsection{Rationale}

%More than 2.17 million people in the Philippines live with visual impairment, which restricts their perception, mobility, navigation, and accessibility to information—depriving them of work opportunities and independence. Wearable SSD is an assistive technology that an individual can wear on their body to supplement a missing sense, helping users navigate both their physical surroundings and digital applications. Additionally, it promotes social interaction that aligns with one of the United Nations' Sustainable Development Goals for health and equality. Supporting local innovation by developing a low-cost, locally made device increases access to technology, thereby enhancing the technology sector. 

%In previous studies, SSDs have been limited in their applicability, as existing devices often perform poorly in real-world situations, encountering cognitive overload problems that overwhelm users with excessive information, making navigation difficult. Another identified concern is that the device design is uncomfortable, as it is bulky and requires extensive training, which limits its practicality. In addition, many devices lack digital integration as capabilities, focusing mainly on obstacle detection and not connecting well with digital tools. These deficiencies underscore the pressing need for higher SSDs that can outperform and enhance existing devices. Therefore, there is a need for a smart navigation hat that improves the functional efficiency and performance of SSDs. This navigation hat integrates AIoT, introducing a smart navigation hat that uses 3D point cloud data to give audio feedback and connects to IoT services for navigation in both physical and digital environments. Furthermore, it enhances sensory processing, focusing on reducing cognitive overload in the system and making navigation simpler for VIP users. Additionally, the hat features customizable options, including a modular operating system that allows users to choose between gesture or voice commands.

Visual Impairment encompasses a spectrum of conditions, from moderate vision loss to complete blindness, which significantly impacts individual’s ability to perceive their environment and perform daily activities (Kumar et al., 2025). This condition also affects the individual’s social interaction and overall quality of life (Que et al., 2025). More than 2.17 million people in the Philippines live with visual impairment, which restricts their perception, mobility, navigation, and accessibility to information—depriving them of work opportunities and independence (Chavarria et al., 2025). To mitigate the challenges encountered by the VIP, assistive technologies and rehabilitation strategies have been developed (Skulimowski, 2025). Traditional assistive aids like white canes offer limited feedback, initiating the development of electronic travel aids that deliver richer environmental information and improve autonomous navigation (Chandra et al., 2025) (Kim, 2024). Another solution is a wearable SSD, an assistive technology that an individual can wear on their body to help users navigate both their physical surroundings and digital applications, and emerging as a potential avenue to restore or gain a sense of spatial perception or awareness and navigational ability (Skulimowski, 2025). Additionally, SSDs promotes social interaction that aligns with one of the United Nations' Sustainable Development Goals for health and equality (Xue et al., 2025).

In previous studies, SSDs have been limited in their applicability, as existing devices often perform poorly in real-world situations, encountering cognitive overload problems that overwhelm users with excessive information, making navigation difficult (Casanova et al., 2025). Another identified concern is that the SSD design is uncomfortable, as it is bulky and requires extensive training, which limits its practicality (Olaosun et al., 2024). In addition, many of these devices lack digital integration as capabilities, focusing mainly on obstacle detection and not connecting well with digital tools (Makati et al., 2024). These deficiencies underscore the pressing need for higher SSDs that can outperform and enhance existing devices. Therefore, a need for development of optimized smart navigation device improves the functional efficiency and performance of SSDs without overlooking the previous concerns like overloading data processing.

In this study the researchers aim to address these matters through the development of a wearable navigation hat integrated with AIoT, which introduces a smart navigation hat that uses 3D point cloud data to give audio feedback and connects to IoT services for navigation in both physical and digital environments. Furthermore, it enhances sensory processing, focusing on reducing cognitive overload in the system and making navigation simpler for VIP users. Additionally, the hat features customizable options, including a modular operating system that allows users to choose between gesture or voice commands.