% Rationale of the study
\section{Rationale}

More than 2.17 million people in the Philippines live with visual impairment, which restricts their perception, mobility, navigation, and accessibility to information—depriving them of work opportunities and independence. Wearable SSD is an assistive technology that an individual can wear on their body to supplement a missing sense, helping users navigate both their physical surroundings and digital applications. Additionally, it promotes social interaction that aligns with one of the United Nations' Sustainable Development Goals for health and equality. Supporting local innovation by developing a low-cost, locally made device increases access to technology, thereby enhancing the technology sector. 

In previous studies, SSDs have been limited in their applicability, as existing devices often perform poorly in real-world situations, encountering cognitive overload problems that overwhelm users with excessive information, making navigation difficult. Another identified concern is that the device design is uncomfortable, as it is bulky and requires extensive training, which limits its practicality. In addition, many devices lack digital integration as capabilities, focusing mainly on obstacle detection and not connecting well with digital tools. These deficiencies underscore the pressing need for higher SSDs that can outperform and enhance existing devices. Therefore, there is a need for a smart navigation hat that improves the functional efficiency and performance of SSDs. This navigation hat integrates AIoT, introducing a smart navigation hat that uses 3D point cloud data to give audio feedback and connects to IoT services for navigation in both physical and digital environments. Furthermore, it enhances sensory processing, focusing on reducing cognitive overload in the system and making navigation simpler for VIP users. Additionally, the hat features customizable options, including a modular operating system that allows users to choose between gesture or voice commands.