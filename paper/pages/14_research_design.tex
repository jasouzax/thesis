\subsection{Research Design/Methods}

This research follows a mixed method design in data collection such as using quantitative data in measuring the accuracy and speed of the device in conveying information, and qualitative data in measuring the comfortability and acclimation of users towards using the device.


% This research implements an experimental and iterative prototyping research focused on developing and testing the wearable AIoT-based visual sensory substition device. This is done by following a development methodology which is similar to the engineering design cycle as seen in figure \ref{fig:researchcycle}.

% \begin{figure}[H]
%     \centering
%     \caption{Development Methodology of the research}
%     \includegraphics[width=\linewidth]{assets/research_cycle.png}
%     \textbf{Source:} \href{https://pressbooks.openeducationalberta.ca/researchasconversationandcommunication/chapter/chapter-1/}{openeducationalberta.ca} 
%     \label{fig:researchcycle}
% \end{figure}


The study to develop a wearable AIoT navigation hat for visually impaired individuals (VIP) will use the Engineering Design Process (EDP), a systematic, iterative framework adapted from sources such as Science Buddies, to guide the project's methodology from problem identification to prototype optimization. This approach ensures a structured path for designing, building, and testing the device, incorporating user feedback and ethical considerations to effectively address VIP needs. 

\begin{figure}[H]
    \caption{Engineering Design Process diagram}
    \includegraphics[width=\linewidth]{assets/edp.png}
    \textbf{Source:} \href{https://www.sciencebuddies.org/science-fair-projects/engineering-design-process/engineering-design-process-steps}{sciencebuddies.org}
    \label{fig:edp}
\end{figure}

\textbf{Identify the Problem and Constraints} \\
Identify the key problems and constraints VIP faces, including difficulties navigating their environment, safety risks, and cognitive overload from existing aids such as white canes and basic sound-sensing devices. Identify the needs for advanced technology that can effectively navigate and replicate human vision without requiring extensive user training. Establish the requirements for the hat and outline the project's necessary functions, including converting 3D environmental data into audio cues and ensuring portability through features such as battery power and ergonomic design. Constraints such as budget considerations, available components (e.g., the Raspberry Pi Zero 2W or other zero-form-factor single-board computers), and ethical guidelines (e.g., obtaining Institutional Review Board (IRB) approval). The focus remains on specific VIP challenges such as social isolation, navigation, mobility, and employment barriers, to align with evaluating AI frame rate (FPS) and user adaptation.

\textbf{Brainstorm Possible Solutions} \\
Illustrate various possible ideas for wearable designs, drawing inspiration from existing technologies such as Sound of Vision and Smart Hat. Consider emphasizing the use of IoT and AI for photogrammetry and inertial measurement units (IMUs), with a creative approach to enhance human senses and thereby reduce cognitive load. The goal is to connect physical and digital experiences to benefit a broader range of users.
Select a Solution and Develop a Design
Select the best design for implementation. This entails creating a detailed development plan that includes component selection, circuit diagrams, prototyping timelines, and considerations for ethical data collection and risk management. The prototype will focus on creating a wearable hat that integrates AIoT, providing audio cues while ensuring comfort and customizability for VIP users.

\textbf{Build a Prototype} \\
In the prototyping stage, the researchers will assemble the components according to specifications and conduct initial tests to verify functionality. Document the entire process, including any challenges faced, to streamline improvements. The development will focus on a user-friendly, efficient, ergonomically designed prototype for VIPs, integrating AIoT features for real-time feedback.

\textbf{Evaluate the Prototype} \\
Evaluation of the prototype involves conducting both quantitative and qualitative tests. This includes measuring AI FPS, depth accuracy, and other collected data, along with user trials to assess comfort and understanding. Feedback from focus groups and careful analysis of results will inform further refinements, helping identify and document any issues encountered during prototype testing.

\textbf{Redesign and Optimize} \\
Based on the feedback received, the prototype design will undergo iterative refinement. Changes may include enhancements to audio algorithms or modifications to extend battery life, ultimately striving to optimize the prototype for practical use by visually impaired individuals.


