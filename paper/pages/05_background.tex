% Background of the study
\subsection{Background of the study}

Visual impairment is a global health problem that significantly affects individual's daily lives by impeding independent navigation, social interaction, and overall quality of life \parencite{theodorou_2023}. There is an estimation of 2.2 billion people globally who are identified as visually impaired and this number could still increase to 2.5 billion by 2050 as stated by the World Health Organization \parencite{who_2023}. In the Philippines, 2.17 million Filipinos are identified as visually impaired as quantized by reports from the Philippines Eye Research Institute (PERI) and Department of Health (DOH) \parencite{shinagawa_2025}. Visual impairment does not pertain to total blindness. According to \cite{who_2023}, visual impairment can be identified and categorized based on the presenting visual acuity
\begin{enumerate}
    \item Mild Vision Impairment, visual acuity is better than 6/18
    \item Moderate Vision Impairment, visual acuity is worse than 6/18 but better than 6/60
    \item Severe Vision impairment, visual acuity is worse than 6/60 but better than 3/60
    \item Blindness, visual acuity is worse than 3/60
    \begin{enumerate}
        \item Blindness with light perception, individuals can only perceive light
        \item Total Blindness, individuals who have no light perception
    \end{enumerate}
\end{enumerate}
Other common types include astigmatism, near-sightedness (myopia) and far-sightedness (hyperopia), which account for a large share of impairments, alongside conditions like cataracts, glaucoma, and age-related macular degeneration (AMD), with AMD alone affecting 8.06 million people globally in 2021 \todo{cite}.

%Visually impaired individuals face safety concerns and challenges such as accidents, falls, and collisions, as well as difficulties with Visually impaired individuals face safety concerns and challenges such as accidents, falls, and collisions, as well as difficulties with navigation, including road crossings and destination location \parencite{ikram_2024,gao_2025,muhsin_2023}. Beyond these practical difficulties, social isolation and reduced access to information further compound the challenges, often limiting educational and employment opportunities \parencite{arvind_2023}. Current existing solutions or aid for visually impaired individuals are guide dogs, white canes, and electronic travel aids \parencite{muhsin_2023}.

% According to \textcite{mishra_2025}, sensory substitution devices (SSDs) are assistive technology that converts information typically perceived through one sensory modality into another, enabling individuals with sensory impairments to access environmental cues they lack. In the context of visual impairment, such devices translate visual data, such as depth or object presence, into haptic or auditory signals to assist the visually impaired in perceiving their environment \parencite{jaiyu_2025}. This intermodal conversion allows users to develop a novel form of perception, effectively substituting the impaired sense with an intact one, enhancing the understanding of the surroundings \parencite{jiang_2025}. These devices could significantly enhance independence by enabling obstacle detection, navigation, and object recognition, particularly for the fully blind who rely on traditional aids like white canes or guide dogs \parencite{tokmurziyev_2025}. However, SSDs are not widely adopted globally due to challenges such as cognitive overload from processing substituted sensory data, extensive training requirements for intuitiveness, ergonomic discomfort in wearables, and processing constraints of non-visual senses like hearing or touch, which have lower bandwidth than vision \parencite{hou_2025} 

Navigation for visually impaired individuals increasingly relies on assistive devices that utilize sensory substitution, converting visual information into auditory or tactile cues to enhance spatial awareness (Skulimowski, 2025). These devices often incorporate IoT sensors to capture real-time environmental data, providing navigational assistance in complex environments (Real \& Araújo, 2023) (Mohamadi et al., 2024). In addition, visually impaired individuals also face safety concerns and challenges such as accidents, falls, and collisions, as well as difficulties with navigation, including road crossings and destination location (Ikram et al., 2024; Gao et al., 2025; Muhsin et al., 2023). Beyond these practical difficulties, social isolation and reduced access to information further compound the challenges, often limiting educational and employment opportunities (Arvind, 2023). Current existing solutions or aid for visually impaired individuals are guide dogs, white canes, and electronic travel aids (Muhsin et al., 2023).

Sensory substitution devices (SSDs) allow users to perceive their environment by converting sensory information from one modality to another, particularly aiding those with visual impairments \parencite{mishra_2025}. These devices help with obstacle detection, navigation, and object recognition, enhancing independence for users relying on traditional aids \parencite{tokmurziyev_2025}. However, global adoption is limited by issues such as cognitive overload, extensive training needs, ergonomic discomfort, and the lower processing bandwidth of non-visual senses compared to vision \parencite{hou_2025}.